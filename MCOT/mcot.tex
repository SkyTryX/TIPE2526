\documentclass{paper}

\usepackage{fancyhdr}
\usepackage[a4paper, total={170mm,257mm}, left=20mm, top=20mm]{geometry}

\begin{document}

\section*{Mise en cohérence des objectifs du TIPE (MCOT) - GIL Dorian}

\section{Positionnements thématiques et mots-clés}
\subsection{Thème}
\begin{enumerate}
\item Informatique
\item Informatique théorique
\item Informatique pratique
\end{enumerate}

\subsection{Mots clés - Key words}
\begin{enumerate}
\item Méthode des tableaux // Method of Analytic Tableaux
\item Logique propositionnelle // Propositional Logic
\item Logique linéaire temporelle // Linear Temporal Logic
\item Satisfiabilité // Satisfiability
\item Vérification de modèle // Model checking
\end{enumerate}

\section{Bibliographie commentée (318/650 mots)}
On souhaite étudier la satisfiabilité de formules logiques dans le contexte de la logique propositionnelle [1]. Des
théorèmes connus montrent que ce problème est NP-complet [1], mais il subsiste la question de
la recherche de la satisfiabilité en temps polynomial. Différents algorithmes ont été imaginés pour résoudre ce
problème en temps exponentiel. Nous nous intéressons notamment à l'une des plus populaires dans la logique
modale : la méthode des tableaux [2].

Une grande problématique dans le domaine de la logique (outre la logique propositionnelle) est la
recherche d'un algorithme permettant de résoudre le problème de satisfiabilité (SAT) en temps polynomial. La
résolution de ce colossal problème impliquerait P = NP [3]. Le but de notre étude n'est évidemment pas de
résoudre ce problème, mais de présenter une approche personnelle du problème de la satisfiabilité en utilisant
la méthode des tableaux et en proposant des implémentations dans le langage OCaml. Dans cette approche,
nous essaierons de trouver des algorithmes qui résolvent une sous-instance de SAT en temps polynomial, et si
ce n'est pas possible, nous trouverons des moyens d'optimiser l'algorithme exponentiel.

Or, il semblerait que la méthode des tableaux soit plus utilisée en logique modale, en particulier dans la logique
linéaire temporelle [4], dans le cadre de la vérification de modèle [5], en particulier dans un algorithme de vérification automatique
en LTL [6]. En effet, dans un monde toujours plus dépendant des différentes technologies dont la création repose sur des algorithmes,
on souhaite vérifier le bon fonctionnement de nos programmes informatiques par rapport aux cahiers des charges associés à un produit.
Nous étudierons ainsi l'application de la méthode des tableaux dans ce cadre en utilisant une implémentation OCaml de la méthode des
tableaux dans la logique linéaire temporelle, tout en proposant un exemple d'objet technique pouvant correspondre à une telle formule.
Cette implémentation de la méthode des tableaux utilise un système de tableau en forme d'arbre, comme on peut le voir traditionnellement
en logique propositionnelle [7].

\section{Problématique retenue}
Comment utiliser la méthode des tableaux pour la résolution de différents problèmes contemporains (en particulier SAT et la vérification de modèle) ?

\section{Objectifs du TIPE}
Les objectifs du TIPE sont d'étudier les différentes utilisations de la méthode des tableaux dans différentes logiques. Nous nous intéresserons en particulier :
\begin{itemize}
    \item À la logique propositionnelle avec une étude de la satisfiabilité d'une famille de formules
    \item À la logique linéaire temporelle avec l'étude d'une formule logique modélisant le fonctionnement attendu d'un objet technologique.
\end{itemize}
Nous nous intéresserons seulement à des exemples précis de formules plutôt qu'au cas général dans le cadre de ce TIPE.

\section{Liste de références bibliographiques} 
\begin{enumerate} 
    \item Pierre Le Barbenchon, Sophie Pichinat, François Schwarzentruber ; Logique : fondements et applications ; Dunod ; ISBN : 978-2-10-082158-7 
    \item Chiara Ghidini (Università di Trento) ; Mathematical Logic: Tableaux Reasoning for Propositional Logic ; https://dit.unitn.it/\verb|~|ldkr/ml2015/slides/PLtableau.pdf 
    \item Garey M.R., Johnson D.S. ; Computers and Intractability: A Guide to the Theory of NP-Completeness, 1 ed.; San Francisco: W. H. Freeman and Company, 1979 
    \item The tableau method for temporal logic: an overview ; Pierre Wolper ;01 Jan 1985 - Logique Et Analyse - Vol. 28, pp 119-136 
    \item Principles of Model Checking ; Christel Baier et Joost-Pieter Katoen ; Chap 5 - 9780262026499 - April 25, 2008 - The MIT Press 
    \item Simple On-the-fly Automatic Verification of Linear Temporal Logic ; Rob Gerth, Doron Peled, Moshe Y. Vardi, Pierre Wolper ; https://orbi.uliege.be/bitstream/2268/116644/1/GPVW95-pstv.pdf 
    \item A traditional tree-style tableau for LTL ; Mark Reynolds ; https://arxiv.org/pdf/1604.03962 - The University of Western Australia ;
\end{enumerate}

\end{document}