\documentclass{paper}

\usepackage{fancyhdr}
\usepackage[a4paper, total={170mm,257mm}, left=20mm, top=20mm]{geometry}

\begin{document}

\section*{Mise en cohérence des objectifs du TIPE (MCOT) - GIL Dorian}

\section{Intro}
\begin{itemize}
    \item \textbf{Titre: }Application de la méthode des tableaux en logique propositionelle et en logique linéaire temporelle
    \item \textbf{Ancrage: }La méthode des tableaux est un algorithme qui renvoie la satisfiabilité d'une formule. Nous étudierons une variante où le tableau aura la forme d'un arbre [7]. Une formule est insatisfaisable ssi toutes les branches ont des cycles (c'est à dire $\phi$ et $\lnot\phi$ sur une même branche)
    \item \textbf{Motivations: }Etant interessé par mes cours de logiques, et après des recherches, j'ai découvert la méthode des tableaux. Après plus de recherches notamment dans ces applications, j'ai découvert et je me suis intéressé à la vérification de modèle, d'où l'étude de la méthode des tableaux dans le cadre de la vérification de modèle.
\end{itemize}

\section{Positionnements thématiques et mots-clés}
\subsection{Thème}
\begin{enumerate}
\item Informatique - Théorique
\item Informatique - Pratique
\item Mathématiques - Autres (Logiques classiques et non-classiques ; Algorithmique)
\end{enumerate}

\subsection{Mots clés - Key words}
\begin{enumerate}
\item Méthode des tableaux // Method of Analytic Tableaux
\item Logique propositionnelle // Propositional Logic
\item Logique linéaire temporelle // Linear Temporal Logic
\item Satisfiabilité // Satisfiability
\item Vérification de modèle // Model checking
\end{enumerate}

\section{Bibliographie commentée (493/650 mots)}
Dans un premier temps, on souhaite étudier la satisfiabilité de formules logiques dans le contexte de la logique propositionnelle. 
Des théorèmes connus montrent que ce problème est NP-complet [1], c'est ainsi que la 
question de la satisfiabilité en temps polynomial est devenue une question à un million de dollars.
Donc une grande problématique dans le domaine de la logique (en outre la logique propositionnelle) est la
recherche d'un algorithme permettant de résoudre le problème de satisfiabilité (SAT) en temps polynomial. La
résolution de ce colossal problème impliquerait en effet P = NP [3]. Différents algorithmes ont été imaginés pour 
résoudre ce problème en temps exponentiel en la taille de la formule étudiée. Nous nous intéressons notamment un de ces
algorithme :la méthode des tableaux.

Dans le cadre de notre étude, cette méthode consiste à construire un arbre avec la formule à la racine, et à 
utiliser des règles pour développer ou créer des branches. On regarde ensuite s'il y a des contradictions dans 
toutes les branches. Si c'est le cas, la formule est insatisfaisable.

Or, il semblerait que la méthode des tableaux soit plus utilisée en logique linéaire temporelle [4], dans le cadre de 
la vérification de modèle, en particulier dans un algorithme de vérification automatique en logique linéaire temporelle [6].
La logique linéaire temporelle est une logique propositionnelle à laquelle on rajoute des opérateurs temporels, ainsi les variables peuvent
avoir des valeurs de vérité qui dépendent du temps, représenté par une suite infinie d'évaluations de vérité des variables. 
La vérification de modèle est un problème dans lequel on souhaite vérifier si le modèle d'un système satisfait une propriété.
La méthode des tableaux est utile dans le cadre d'une méthode utilisant les automates de Büchi pour créer un modèle de notre système
ainsi que des formules logiques modélisant les propriétés que notre modèle doit vérifier. Nous ferons abstraction des algorithmes
derrière et nous admettrons dans le cadre de ce TIPE que la méthode des tableaux est grandement utilisée dans le cadre de la vérification 
de modèle. [5]

C'est ainsi que, dans un monde toujours plus dépendant des différentes technologies dont la création repose sur des algorithmes, on souhaite 
vérifier le bon fonctionnement de nos programmes informatiques par rapport aux cahiers des charges associés à un produit.
Nous étudierons ainsi l'application de la méthode des tableaux dans ce cadre en utilisant une implémentation OCaml de la 
méthode des tableaux dans la logique linéaire temporelle. Nous utiliserons ce que nous avons appris pour résoudre un exemple concret
en lien avec la vérification de modèle.

\section{Problématique retenue}
Comment utiliser la méthode des tableaux pour la résolution du problème SAT et de la vérification de modèle ?

\section{Objectifs du TIPE}
Les objectifs du TIPE sont d'étudier les différentes utilisations de la méthode des tableaux dans différentes logiques. Nous nous intéresserons en particulier :
\begin{itemize}
    \item À la logique propositionnelle avec une étude de la satisfiabilité d'une famille de formules.
    \item À la logique linéaire temporelle avec l'étude d'une formule logique modélisant le fonctionnement attendu d'un objet technologique.
\end{itemize}


\section{Liste de références bibliographiques} 
\begin{enumerate} 
    \item Pierre Le Barbenchon, Sophie Pichinat, François Schwarzentruber ; Logique : fondements et applications ; Dunod ; ISBN : 978-2-10-082158-7 
    \item Chiara Ghidini (Università di Trento) ; Mathematical Logic: Tableaux Reasoning for Propositional Logic ; https://dit.unitn.it/\verb|~|ldkr/ml2015/slides/PLtableau.pdf 
    \item Garey M.R., Johnson D.S. ; Computers and Intractability: A Guide to the Theory of NP-Completeness, 1 ed.; San Francisco: W. H. Freeman and Company, 1979 
    \item The tableau method for temporal logic: an overview ; Pierre Wolper ;01 Jan 1985 - Logique Et Analyse - Vol. 28, pp 119-136 
    \item Principles of Model Checking ; Christel Baier et Joost-Pieter Katoen ; Chap 5 - 9780262026499 - April 25, 2008 - The MIT Press 
    \item Simple On-the-fly Automatic Verification of Linear Temporal Logic ; Rob Gerth, Doron Peled, Moshe Y. Vardi, Pierre Wolper ; https://orbi.uliege.be/bitstream/2268/116644/1/GPVW95-pstv.pdf 
    \item A traditional tree-style tableau for LTL ; Mark Reynolds ; https://arxiv.org/pdf/1604.03962 - The University of Western Australia ;
\end{enumerate}

\end{document}