\documentclass{paper}

\usepackage{fancyhdr}
\usepackage[a4paper, total={170mm,257mm}, left=20mm, top=20mm]{geometry}

\begin{document}

\section*{Mise en Cohérence des Objectifs du TIPE (MCOT) - GIL Dorian}

\section{Positionnements thématiques et mots-clés}
\subsection{Thème}
\begin{enumerate}
    \item Informatique
\end{enumerate}

\subsection{Mots clés - Key words}
\begin{enumerate}
    \item Méthode des tableaux - Method of analytic tableaux
    \item Satisfaisabilité - Satisfiability
    \item Logique Propositionelle - Propositional Logic
    \item OCaml - OCaml
    \item Classe de Complexité - Complexity class
\end{enumerate}

\section{Bibliographie commentée}
On souhaite étudier la satisfiabilité de formules logiques dans le contexte de la logique propositionnelle [1].  Des théorèmes connus montrent que ce
problème est résoluble en temps exponentiel, mais il subsiste la question de la recherche de la satisfiabilité en temps polynomial. Différents
algorithmes ont été imaginés pour résoudre ce problème en temps exponentiel. Nous nous intéressons notamment à l'une des plus populaires dans la
logique modale : la méthode des tableaux [2]. Nous concentrerons son étude néanmoins dans la logique propositionnelle qui est la racine de la logique 
modale.

Une grande problématique dans le domaine de la logique (en outre de la logique propositionnelle) est la recherche d'un algorithme permettant de 
résoudre le problème de satisfiabilité (SAT) en temps polynomial. La résolution de ce colossal problème impliquerait $P=NP$ [3]. Le but de notre étude
n'est évidemment pas de résoudre ce problème, mais de présenter une approche personnelle du problème de la satisfiabilité en utilisant la méthode des
tableaux et en proposant des implémentations dans le langage OCaml. Dans cette approche, nous essayerons de trouver des algorithmes qui résolvent une
sous-instance de SAT en temps polynomial, et si ce n'est pas possible, nous trouverons des moyens d'optimiser l'algorithme exponentiel.

\section{Problématique retenue}
Comment implémenter et optimiser la méthode des tableaux dans la logique propositionnelle pour résoudre le problème SAT ?

\section{Objectifs du TIPE}
À l'issue de l'étude, je souhaite proposer plusieurs études complètes de formules logiques. Ces études contiennent:
\begin{itemize}
    \item Un sous-ensemble de formules
    \item Un algorithme (polynomial ou pas) et son implémentation résolvant SAT pour ce type de formule avec la démarche scientifique de recherche derrière l'algorithme.
    \item Une comparaison avec d'autres algorithmes connus
    \item Une preuve de terminaison et de correction
\end{itemize}
Et la somme permet de proposer, comme d'annoncer, une approche personnelle du problème SAT, avec la méthode des tableaux, et à partir de nos recherches, 
de déduire ces avantages et ces inconvénients.

\section{Liste de références bibliographiques}
\begin{enumerate}
    \item Pierre Le Barbenchon, Sophie Pichinat, François Schwarzentruber ; Logique: fondements et applications ; Dunod ; ISBN: 978-2-10-082158-7
    \item Chiara Ghidini (Università Di Trento) ; Mathematical Logic: Tableaux Reasoning for Propositional Logic ; https://dit.unitn.it/\textasciitilde ldkr/ml2015/slides/PLtableau.pdf
    \item Garey M.R., Johnson D.S. ; Computers and Intractability: A Guide to the Theory of NP-Completeness, 1 ed.; San Francisco: W. H. Freeman and Company, 1979
\end{enumerate}

\end{document}