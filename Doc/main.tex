\documentclass{paper}

\usepackage{amsmath,amssymb,amsfonts, listings, fancyhdr, stmaryrd, array}
\usepackage[many]{tcolorbox}
\usepackage[a4paper, total={170mm,257mm}, left=20mm, top=20mm]{geometry}
\newtheorem{prop}{Proposition}
\newtheorem{defi}{Definition}
\newcolumntype{C}{>$c<$}
\tcolorboxenvironment{prop}{enhanced, borderline={0.8pt}{0pt}{blue}, borderline={0.4pt}{2pt}{cyan}, boxrule=0.4pt, colback=white, coltitle=black, sharp corners}
\tcolorboxenvironment{defi}{ enhanced, borderline={0.8pt}{0pt}{red}, borderline={0.4pt}{2pt}{orange}, boxrule=0.4pt, colback=white, coltitle=black, sharp corners}

\pagestyle{fancy}
\fancyhead[C]{TIPE 25/26 - Dorian GIL}

\begin{document}
\setlength{\headheight}{13.07225pt}
\addtolength{\topmargin}{-1.07225pt}



\section*{Piste de recherche de sujet}
\subsection*{Decomposition de graphe}
07/02
Premiere idée de sujet: Identification de graphe décomposable en cycle hamiltonien en temps polynomial

\subsection*{Logique, systeme de preuve}
07/02
J'aimerai néanmoins plus m'orienté vers la logique, piste de recherche: Méthode des tableaux, Zenon theorem prover, calcul séquent.
Exploration de la méthode des tableaux: J'ai besoin de trouové un livre présentant cette méthode

\section*{Méthode des tableaux}

\begin{defi}[Modèle]
    Un \textit{modèle} d'une formule $\phi$ est une valuation qui rend vraie cette formule. On note l'ensemble des modèles de $\phi$ par:
    $$Mod(\phi) := \{v\in Val | v \vDash \phi\}$$
    $Val$ étant l'ensemble des valuations de $phi$ et $v \vDash \phi$ signfiant que la valuation $v$ satisfait $\phi$
\end{defi}
\begin{defi}[Conséquence Logique]
    Une formule $\phi$ est \textit{conséquence logique} d'une formule, notée $\psi$ si $Mod(\psi) \subseteq Mod(\phi)$. On note cela $\psi \vDash \phi$
\end{defi}
On note $n\in [ 1,+\infty ] $
\begin{defi}
    \textit{La méthode des tableaux} consistent en prouvant une assertion $B$ ayant pour hypothèse $(A_n)$ en montrant
    que $\{A_1,\dots,A_n, \lnot B\}$ est insatisfaisable.
\end{defi}

Le procédé consiste en 
\end{document}